%%% A template for a simple PDF/A file like a stand-alone abstract of the thesis.

\documentclass[12pt]{report}

\usepackage[a4paper, hmargin=1in, vmargin=1in]{geometry}
\usepackage[a-2u]{pdfx}
\usepackage[utf8]{inputenc}
\usepackage[T1]{fontenc}
\usepackage{lmodern}
\usepackage{textcomp}

\begin{document}
Using convolutional neural networks (CNN) for high dimensional data such as images has shown some promising results, but it has issues with features without spatial correlation. I test a streamlined version of the REFINED (REpresentation of Features as Images with NEighborhood Dependencies) model and compare its performance on a ligand binding site detection dataset. I find no significant evidence supporting that this method has better predictive power than state-of-the-art approaches or CNNs using matrices with randomly assigned feature positions.

\noindent\rule{2cm}{0.4pt}

Použití konvolučních neuronových sítí (CNN) pro vysokorozměrná data, jako jsou obrázky, ukázalo některé slibné výsledky, ale má problémy s featurami bez prostorové korelace. Testuji zjednodušenou verzi modelu REFINED (REpresentation of Features as Images with Neighborhood Dependencies) a porovnávám jeho výkon na datovém souboru detekce vazebného místa ligandu. Dále se pokusím najít korelaci mezi silou CNN a použitou metodou pro transformaci vstupních vektorů do obrázků. Nenašel jsem žádný výsledek, který by podporoval, že by REFINED mělo lepší prediktivní sílu než nejmodernější přístupy nebo jenom CNN využívající matice s náhodně přiřazenými pozicemi prvků.


\end{document}
