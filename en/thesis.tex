%%% The main file. It contains definitions of basic parameters and includes all other parts.

%% Settings for single-side (simplex) printing
% Margins: left 40mm, right 25mm, top and bottom 25mm
% (but beware, LaTeX adds 1in implicitly)
\documentclass[12pt,a4paper]{report}
\setlength\textwidth{145mm}
\setlength\textheight{247mm}
\setlength\oddsidemargin{15mm}
\setlength\evensidemargin{15mm}
\setlength\topmargin{0mm}
\setlength\headsep{0mm}
\setlength\headheight{0mm}
% \openright makes the following text appear on a right-hand page
\let\openright=\clearpage

%% Settings for two-sided (duplex) printing
% \documentclass[12pt,a4paper,twoside,openright]{report}
% \setlength\textwidth{145mm}
% \setlength\textheight{247mm}
% \setlength\oddsidemargin{14.2mm}
% \setlength\evensidemargin{0mm}
% \setlength\topmargin{0mm}
% \setlength\headsep{0mm}
% \setlength\headheight{0mm}
% \let\openright=\cleardoublepage

%% Generate PDF/A-2u
\usepackage[a-2u]{pdfx}

%% Character encoding: usually latin2, cp1250 or utf8:
\usepackage[utf8]{inputenc}

%% Prefer Latin Modern fonts
\usepackage{lmodern}

%% Further useful packages (included in most LaTeX distributions)
\usepackage{amsmath}        % extensions for typesetting of math
\usepackage{amsfonts}       % math fonts
\usepackage{amsthm}         % theorems, definitions, etc.
\usepackage{bbding}         % various symbols (squares, asterisks, scissors, ...)
\usepackage{bm}             % boldface symbols (\bm)
\usepackage{graphicx}       % embedding of pictures
\usepackage{fancyvrb}       % improved verbatim environment
\usepackage{natbib}         % citation style AUTHOR (YEAR), or AUTHOR [NUMBER]
\usepackage[nottoc]{tocbibind} % makes sure that bibliography and the lists
			    % of figures/tables are included in the table
			    % of contents
\usepackage{dcolumn}        % improved alignment of table columns
\usepackage{booktabs}       % improved horizontal lines in tables
\usepackage{paralist}       % improved enumerate and itemize
\usepackage{xcolor}         % typesetting in color
\usepackage{hyperref}
\usepackage{listings}
\usepackage{color}
\usepackage{dirtytalk}
\usepackage{caption}
\usepackage{subcaption}
\usepackage[printonlyused,withpage]{acronym}

\DeclareRobustCommand{\bbone}{\text{\usefont{U}{bbold}{m}{n}1}}

\DeclareMathOperator{\EX}{\mathbb{E}}% expected value

\newcommand{\distras}[1]{%
  \savebox{\mybox}{\hbox{\kern3pt$\scriptstyle#1$\kern3pt}}%
  \savebox{\mysim}{\hbox{$\sim$}}%
  \mathbin{\overset{#1}{\kern\z@\resizebox{\wd\mybox}{\ht\mysim}{$\sim$}}}%
}

\definecolor{dkgreen}{rgb}{0,0.6,0}
\definecolor{gray}{rgb}{0.5,0.5,0.5}
\definecolor{mauve}{rgb}{0.58,0,0.82}

\lstset{frame=tb,
  language=Python,
  aboveskip=3mm,
  belowskip=3mm,
  showstringspaces=false,
  columns=flexible,
  basicstyle={\small\ttfamily},
  numbers=none,
  numberstyle=\tiny\color{gray},
  keywordstyle=\color{blue},
  commentstyle=\color{dkgreen},
  stringstyle=\color{mauve},
  breaklines=true,
  breakatwhitespace=true,
  tabsize=3
}
%%% Basic information on the thesis

% Thesis title in English (exactly as in the formal assignment)
\def\ThesisTitle{Using convolutional neural networks to detect protein-ligand binding sites}

% Author of the thesis
\def\ThesisAuthor{Samuel Soukup}

% Year when the thesis is submitted
\def\YearSubmitted{2024}

% Name of the department or institute, where the work was officially assigned
% (according to the Organizational Structure of MFF UK in English,
% or a full name of a department outside MFF)
\def\Department{Department of Software Engineering}

% Is it a department (katedra), or an institute (ústav)?
\def\DeptType{Department}

% Thesis supervisor: name, surname and titles
\def\Supervisor{doc. RNDr. David Hoksza, Ph.D.}

% Supervisor's department (again according to Organizational structure of MFF)
\def\SupervisorsDepartment{Department of Software Engineering}

% Study programme and specialization
\def\StudyProgramme{Computer Science (B0613A140006)}
\def\StudyBranch{Artificial Intelligence Bc. (NIUI19B)}

% An optional dedication: you can thank whomever you wish (your supervisor,
% consultant, a person who lent the software, etc.)
\def\Dedication{
Computational resources were provided by the e-INFRA CZ project (ID:90254),
supported by the Ministry of Education, Youth and Sports of the Czech Republic.
}

% Abstract (recommended length around 80-200 words; this is not a copy of your thesis assignment!)
\def\Abstract{%
Using convolutional neural networks (CNN) for high dimensional data such as images has shown some promising results, but it has issues with features without spatial correlation. I test a streamlined version of the REFINED (REpresentation of Features as Images with NEighborhood Dependencies) model and compare its performance on a ligand binding site detection dataset. I explore the correlation between the CNN performance and the technique used to transform input vectors into images. I find no significant evidence supporting that REFINED has better predictive power than state-of-the-art approaches or CNNs using matrices with randomly assigned feature positions.
}

\def\AbstractCS{%
Použití konvolučních neuronových sítí (CNN) pro vysokorozměrná data, jako jsou obrázky, ukázalo některé slibné výsledky, ale má problémy s featurami bez prostorové korelace. V této práci jsem otestoval zjednodušenou verzi modelu REFINED (REpresentation of Features as Images with Neighborhood Dependencies) a porovnával její výkon na datasetu detekce vazebného místa ligandu. Dále jsem se pokusil najít korelaci mezi prediktivní silou CNN a použitou metodou pro transformaci vstupních vektorů do obrázků. Nenašel jsem žádný výsledek, podporující hypotézu, že by REFINED mělo lepší prediktivní sílu než nejmodernější přístupy nebo jenom CNN využívající matice s náhodně přiřazenými pozicemi prvků.
}
% 3 to 5 keywords (recommended), each enclosed in curly braces
\def\Keywords{%
{protein} {bioinformatics} {machine learning} {cnn}
}
\def\ThesisKeywordsCS{%
{protein} {bioinformatika} {strojové učení} {cnn}
}

%% The hyperref package for clickable links in PDF and also for storing
%% metadata to PDF (including the table of contents).
%% Most settings are pre-set by the pdfx package.
\hypersetup{unicode}
\hypersetup{breaklinks=true}

% Definitions of macros (see description inside)
\include{macros}


% Title page and various mandatory informational pages
\begin{document}

\include{title}

%%% A page with automatically generated table of contents of the bachelor thesis

\tableofcontents

%%% Each chapter is kept in a separate file
\chapter*{Introduction}
\addcontentsline{toc}{chapter}{Introduction}

In recent years, \textit{machine learning} (ML) has been used in many applications, and with all of this success, large datasets have been collected and can be analyzed. For instance, bioinformatics and cheminformatics can gather enormous datasets through high-throughput methods, allowing us to understand genomic data better than ever before (e.g., \cite{NCI-DREAM}). However, these datasets often have many features compared to the number of samples. This often necessitates feature selection before the data is used in a predictive model. A high-performance predictive model with a feature engineering mechanism could bring high value in similar instances.

One such algorithm - REFINED (REpresentation of Features as Images with NEighborhood Dependencies), proposed by \cite{REFINED} converts the high-dimensional data into images and uses a \textit{convolutional neural network} as a predictor. In this work, I explore a streamlined version of this algorithm and compare it to state-of-the-art approaches.

For this approach to make sense, it requires high-dimensional data (an image with a resolution of 3x3 values doesn't make much sense for a CNN). Because of this, I test this approach on data generated using P2Rank by \cite{P2RANK}, which is in the form of a point cloud. The feature vector can be extended by appending the features from the N nearest points. This allows me to create datasets with an arbitrary amount of features. Also, this dataset is from a similar domain to the one used in the original paper, making it the perfect candidate. Lastly, a part of the P2Rank article is also training a ranking model, which allows me to set a baseline for the models.

The simplest way to get an image from a vector is to reshape the vector into a matrix (a black-and-white image in practice). However, this simplest way could be improved by rearranging the pixels into a more advantageous defined order.

The REFINED algorithm's core takes a matrix with randomly distributed features. It creates gradients (i.e., minimizes local contrast) in the image by minimizing the mean difference between adjacent pixels from all data points. This approach is visualized in figure~\ref{fig:whole-visualisation} and discussed in greater detail in the corresponding section.

\begin{figure}
    \centering
    \includegraphics[width=1\linewidth]{whole_pipeline_visualization.png}
    \caption{The general REFINED pipeline. 1) Point cloud, from which features are collected. 2) Long (size $> 900$) feature vector describing a data point. 3) A matrix created by randomly assigning features to positions in the matrix. 4) A REFINED matrix showing gradients created by moving feature positions. 5) CNN model used as a predictor.}
    \label{fig:whole-visualisation}
\end{figure}

\chapter{ML Models}

In this thesis, I have implemented and tested 3 various models:
\begin{enumerate}
    \item Baseline - Random forest classifier
    \item Random forest classifier on surroundings
    \item REFINED: https://doi.org/10.1038/s41598-021-90923-y
    
\end{enumerate}

\section{Model interface}

All of these models have a common interface of a method  \texttt{predict(protein)}, which takes a protein as an input and returns the predicted probability of each input point being a ligand binding site.

\subsection{Input protein}

The input protein is a 3D cloud of points with features, where each point represents an atom [TODO: TRUE?] in the protein and features characterize chemical and biological characteristics of the described area.

In the program, this is represented by a 2D array of features by locations. The physical position is one of the features, using which we can recreate the points cloud.

\subsection{Output probabilities}

The output of the models is the predicted probability of each location to be a ligand binding site. Using the input order, the output is simply an array of predicted probabilities.

\section{Baseline Random Forest Model}

As a baseline model, we are using a random forest classifier as described in the original paper. It does training and prediction point-wise.

This means that it predicts the probability of a site being ligand binding by only the features logged for the given site.

\section{Surroundings extraction}

For the following models, we need to have features for the surroundings of the site as well as the features logged for it.

To achieve this, we run a simple quadratic algorithm for finding the k nearest neighbors. Then we just append their features to the original ones. For the following algorithms, we use these features instead of the original ones.

\section{Random Forest Model on surroundings}

This model uses RFC as well, but it uses not only the given site's features but also features of the surrounding \textit{k} sites, extracted as described above[TK, link].

\section{REFINED}
%  https://www.nature.com/articles/s41598-021-90923-y#citeas


% \chapter{Used data}

\section{Raw data intuition}

The raw input into any of our models is the protein's 3D structure. This is represented by a point cloud on the protein's surface. Each point has a feature vector attached, which represents the chemical properties of the given accessible surface patch. Each point also has a ligand-binding class, which the model is used to train and then predict. These can be seen in the figure~\ref{fig:data_point_cloud}.

\begin{figure}
    \centering
    \includegraphics[width=1\linewidth]{img/protein.png}
    \caption{Point cloud with 2 highlighted ligand-binding sites}
    \label{fig:data_point_cloud}
\end{figure}

In the implementation, these points are represented as tabular data, with three columns representing the X, Y, and Z axis, respectively. From this data, the point cloud can be recreated.

\section{Features}

\section{Surroundings extraction}

For the following models, we need to have features for the site's surroundings and the features logged for it.

To achieve this, we run a simple quadratic algorithm for finding the k nearest neighbors. Then, we append their features to the original ones. We use these features instead of the original ones for the following algorithms.

\chapter{Methods}

In this chapter,  firstly, I'm going to describe the interface each method needs to fulfill for a better comparison. Then, I describe the individual models. In the end, I describe the metrics used for comparing the 
\section{Models}
\subsection{Model interface}

All of these models have a common interface of a method  \texttt{predict(protein)}, which takes a protein as an input and returns the predicted probability of each input point being an LBS. For better performance, if not specified otherwise, each model accepts the protein in the surroundings-based format as described in \hyperref[Surroundings]{the corresponding section}. This way, the surroundings dataset can be cached, as the computation of it takes multiple hours.


\subsection{Baseline - small Random Forest Model}
As the first model, we recreate the original model from \cite{P2RANK}. This is the only model that takes the data in the original format. We're using this as a baseline model to compare other models, with the one proposed in the original paper.

This model uses scikit-learn's RandomForestClassifier (RFC) [\cite{scikit-learn}] as a predictor. As described in the original paper, it uses 200 trees with no depth limit and considers a maximum of 6 features in each split. The rest of the parameters are set to default values as described in the \hyperlink{https://scikit-learn.org/1.1/modules/generated/sklearn.ensemble.RandomForestClassifier.html}{corresponding documentation}.


\subsection{Random Forest Model}

Similarly to the Baseline RF model, this model uses an RFC as the predictor. As with all the following models, this one is trained and used on the surroundings dataset. Hyperparameters were tuned using Randomized Parameter Optimization (RPO) from \cite{scikit-learn} with the goal of maximizing the accuracy on a 0.2 validation split. The best hyperparameters found were Tk.

\subsection{PCA + RFC}

Even though RFCs with some modifications can be used for high-dimensional data (such as in \cite{Genuer}), a common approach for handling this case is applying some DM technique as a preprocessing step, most commonly PCA. This is one of the models tested as a baseline to ensure that RFC will not provide falsely low results. PCA's target dimension was handled as a hyperparameter. Using RPO, the best hyperparameters for accuracy on the 0.2 validation set found were Tk.

\subsection{Dense neural network}

Another common approach for classification is dense neural networks (NN). The exact architecture was created by hyperparameter tuning using the Keras Tuner framework towards a 0.2 validation split. The model was optimized using Adam to minimize binary cross-entropy on a 1-unit, sigmoid-activated last layer.

\subsection{REFINED CNN}

This model simplifies an approach proposed by \cite{REFINED}. Compared to the original approach, the DR preprocessing step is skipped. This approach was not tested in the original paper, as some DR method was used in every experiment. The whole algorithm is simplified in the following list. Because this approach is not well known, I'll describe the details in the text following this list. You can see a high-level comparison with the original paper in Figure Tk.

\begin{enumerate}
    \item Input: Set of samples $X = x_1, ... x_n$, where $\forall i x_i \in \mathbb{R}^{D}$, where $D$ is the number of features for each sample (in this case feature vector as described in \hyperref[Surroundings]{the surroundings extraction section}.
    \item We regulize each feature to follow $f \sim N(0,1)$
    \item Reshape each sample to $x_i \in R^{k\times l}$, where $k\cdot l = D$.
    \item We reorder features in the images to create gradients.
    \item Using this permutation, we create a matrix for each sample used in the following step.
    \item We train a CNN on the provided data.
\end{enumerate}

\subsubsection{Pre-REFINED preprocessing}

The first models trained with this approach didn't use any preprocessing. I decided not to recreate the BMDS (or other DR methods) mentioned in the original article. But there was an important part of this step that I didn't account for - feature normalization. This caused issues in the REFINED core and later in the CNN predictions.

Non-normalized data made results from REFINED core visually worse than normalized ones. After explaining the relevant context, I'll discuss why this is the case both in the REFINED core and CNN sections.

After normalization, features are randomly permutated to make this method perform independently in the order of the input features. Then, the vector is reshaped into a matrix of the desired dimensions. 

\subsubsection{REFINED core}

Now comes the interesting part of the algorithm. To allow the location of the feature in the image to have some meaning, we try to minimize over all the permutations of features the value function of 
    $$ f(O|X) = \sum_{s=1}^{n} \sum_{i, j, k, l =1}^n (x^O_{s, ij} - x^O_{s, kl})^2 \cdot \abs{[i,j] - [k,l]}^{-1}$$
Where
\begin{itemize}
    \item $P([D])$ is the set of all permutations on the list of features from $X$,
    \item $F(O|X): P([D]) \rightarrow \mathbb{R}$ is the value function,
    \item $O \in P([D])$ is an ordering of the features
    \item $x^O_s$ is the sample $x_s$ ordered by the ordering $O$
\end{itemize}
Then $x^O_{s, ij}$ is the value is sample $x_s$ on matrix index $[i,j]$ given that the features are ordered by the ordering $O$.

As this is computationally an exponential problem, we approximate this optimum using the hill climbing algorithm (HCA). HCA is not guaranteed to find the optimal permutation. Still, it works well enough as it converges to some local minimum and for convex functions to the global one. I expect that this value function is almost convex (any local minimum will be relatively close to the global one). There isn't a technically viable solution to find the global minimum, as this would require calculating the value function for each permutation (which is $D!$ - or for 900 features $>6\cdot 10^{2269}$). This method was also used in the original paper, so I'm reusing it here.

HCA requires not only the value function $f(X|O)$ but also the neighborhood function $h(O): P([D]) \rightarrow X \in \mathcal{P}(P([D]))$ (output is a subset of $P([D])$ - or a list of orderings). This can be easily computed with the following algorithm:
\begin{lstlisting}
def h(ordering:np.ndarray) -> List[np.ndarray]:
  ret = []
  for i in range(ordering.shape[0]):
    for j in range(ordering.shape[1]):
      for (k, l) in neighboring_pixels(i,j):
        ret.append(swapped(ordering, (i, j), (k, l)))
  return ret

def swapped(array: np.ndarray,
        index0: Tuple[int, int],
        index1: Tuple[int, int]) -> np.ndarray:
    array[index0], array[index1] = array[index1], array[index0]
    return array

def neighboring_pixels(x: int, y: int) -> List[Tuple[int, int]]:
    return [(x + dx, y + dy) 
              for dx in (-1, 0, 1)
              for dy in (-1, 0, 1)
              if (0 <= x + dx < k and 0 <= y + dy < l)
                or not (dx == dy == 0)
            ]
\end{lstlisting}

For complexity's sake, all the neighbors aren't calculated together for the whole ordering, but after each pixel's neighbors are calculated (so for each $i,j$ from the code), we greedily choose the ordering with the lowest value function (or keep the original one, if it is the best) and calculate all the remaining neighbors from this modified state. This increases performance, as we can make as many as $D$ steps per calculation of value functions for the whole ordering neighborhood. But, as it follows the value function's gradient more approximately, it could lead to a higher chance of ending up in a local minimum. Still, this algorithm runs for multiple hours as is, and this can be partly avoided by running multiple HCA's in parallel.

For this work, I used my own Python implementation of HCA to ensure a smoother integration into the rest of the pipeline. Some more tricks were used to improve the speed of the computations, such as using a Numba JIT compiler, or caching value function results. But still, I expect its speed to be lower than some available libraries using C code as the backend.

After this algorithm is finished, we have a feature ordering with correlated features near each other in a generated image and negatively correlated features further away from each other. In Figure Tk., the process of how features are moving can be seen. Also, this can make it easier for humans to see differences between samples compared to looking at the data simply as a vector.

In one of the first experiments, I tried using a genetic evolutionary algorithm (GEA), but it proved to be much slower than the simple HCA - at least with the configuration used then. HCA can be further optimized and run multiple times in parallel, avoiding its pitfalls (getting stuck in a local minimum). Although I didn't succeed with GEA in this part of the algorithm, with some additional work, it might be faster. But this is out of the scope of this work.

As foreshadowed earlier, the REFINED core does not work well with non-normalized data. This can be easily explained. The REFINED core works by aligning correlated features together. However, to make the process easier, only the $L2$ norm of the difference between each feature is computed. This causes features with different means to be further away from each other. Not only that, it causes features with higher variations to be further away from each other. Small differences between low-variational features then influence the value function very little, compared to high-variational features with vastly different means. However, this is not useful information for the predictor following this step.


%  https://www.nature.com/articles/s41598-021-90923-y#citeas
\section{Evaluation Criteria}

Every method was evaluated in terms of speed and performance criteria. In the following part, I will describe the process of gathering these metrics.

All time-related metrics were gathered on an AMD Ryzen 7 PRO 5850U without GPU acceleration.

\subsection{Training Wall Time}

The training wall time measures the wall time for training a model with already set hyperparameters. We do not include the hyperparameter tuning as for different models, different spaces need to be covered, and a different framework is used for optimizing RFC models and Neural Network-based ones.

The time counter starts just before initializing the untrained model and ends directly after training has ended. This means that evaluation of the model on the test set is not included in this metric, but validation during the training is (e.g., per epoch validation in (C)NN training).

\subsection{Inference Wall Time}

As training is done only once per model lifespan, but inference happens on every model usage, inference time could be much more important depending on the use case.

This time is measured during test-set evaluation and measures only the time the model runs on the whole dataset. This doesn't include metric evaluation of the results, model loading into memory, or any other task.

\subsection{Model performance metrics}

Because all used datasets are imbalanced (97 \% of samples are negative), the f1-score is used as the main evaluation metric. However, accuracy, precision, recall, and false positive rate (FPR) are all calculated. Given that $TP$ is the number of true positives, $FP$ false positives, $TN$ true negatives and $FN$ false positives, aforementioned metrics are defined as follows:
$$F_1 = \frac{2TP}{2TP + FP + FN}$$
$$Accuracy = \frac{TP + TN}{TP + FP + TN + FN}$$
$$Precision = \frac{TP}{TP + FP}$$
$$Recall = \frac{TP}{TP + FN}$$
$$FPR = \frac{FP}{FP + TN}$$

For each dataset, a per protein value of each metric is calculated. Then, a 95 \% interval is created for the mean value. These are then plotted to show a comparison.
\chapter{Experiments}

\section{Overall comparison}

The main experiment consists of comparing all the methods in their whole pipeline. First, I run hyperparameter optimization for each method as described in its description. Then, with these hyperparameters, a model is trained. The \textit{chen11} dataset is used for training, and metrics are calculated on the \textit{coach420} dataset.

With this experiment, I want to test whether REFINED performs better on this dataset than other state-of-the-art approaches. 
\begin{figure}
    \centering
    \includegraphics[width=0.5\linewidth]{img/Accuracy.png}
    \caption{Accuracy measurements for all models}
    \label{fig:accuracy}
\end{figure}
\begin{figure}
    \centering
    \includegraphics[width=0.5\linewidth]{F1_score.png}
    \caption{F1 scores for all models}
    \label{fig:f1}
\end{figure}

\begin{figure}
    \centering
    \includegraphics[width=0.5\linewidth]{Precision.png}
    \caption{Precision for all models}
    \label{fig:precision}
\end{figure}

\begin{figure}
    \centering
    \includegraphics[width=0.5\linewidth]{Recall.png}
    \caption{Recall for all models}
    \label{fig:recall}
\end{figure}

As can be seen in the attached figures (~\ref{fig:accuracy}, ~\ref{fig:precision}, ~\ref{fig:f1}, ~\ref{fig:recall}) REFINED model is on par with state-of-the-art approaches, but does not bring higher performance as reported in \cite{REFINED}. However, it shows better results than Random CNN and even Random Normalized CNN.

\section{REFINED progression test}

With the negative results of the previous test, I wanted to test whether training a CNN on REFINED-transformed images produces better results than on images with randomly allocated positions. I have taken the individual epochs of REFINED and trained a CNN model on it to test this. The models are called \texttt{CNN-i}, where \texttt{i} is the number of epochs of REFINED used as input. So \texttt{CNN-0} will have features distributed very closely to randomly and \texttt{CNN-43} is a fully trained REFINED model. All of these models were trained using CNN hyperparameters found for the REFINED model in the first experiment.

Then, I tested whether there was any correlation between the REFINED score and CNN's performance. The obvious metric to compare is the validation loss, as it represents the training goals the best.

\begin{figure}
    \centering
    \includegraphics[width=0.5\linewidth]{progression.png}
    \caption{Best CNN validation loss plotted against }
    \label{fig:REFINED_progression}
\end{figure}

In ~\ref{fig:REFINED_progression} can be seen that there is little to no correlation between the REFINED score and the model's performance.

\section{REFINED transformation visualized}
At the beginning of this thesis, one of the motivations for this work was the interest in seeing high-dimensional data visualised in some manner. And REFINED does achieve that goal. 
\chapter{Discussion}

\section{REFINED as a predictor}
In this work, I present an algorithm using a \ac{CNN} for high-dimensional data. \cite{REFINED} proposed a more complicated method, which has shown better performance than other state-of-the-art approaches. The REFINED algorithm uses a CNN, which has been shown to improve results mainly on image-based tasks (such as \cite{AlexNet}). However, when tested on the used datasets, I have not seen any significant improvements compared to other methods.

In the second experiment, I tried to find a correlation between a lower (better) REFINED score and better CNN performance. However, I have not detected any correlation between these two factors.

Compared to the original paper, the used dataset has more samples, which could cause a diminishing result of the regularization effect provided by the CNN architecture. However, through hyperparameter optimization, a comparably large model was found (\textit{NN} has 36k parameters, compared to the 1.8 M parameters in \textit{REFINED}'s CNN). This could have been caused by the hyperparameter space being too large, the smaller models not being explored, or the larger models performing slightly better \footnote{Based on other runs, I'm inclined to believe that it is the case that the smaller models are very similar in performance. Check the section ~\ref{metrics_availability}, where the availability of some more models is described.}.

With all of these experiments, I tried to find some way of showing that REFINED works or that it, for some reason, makes sense to go through all the trouble to rearrange the features to obtain some benefit. But I wasn't able to do that. With all of this, I do not see a strong reason why such REFINED should outperform a simple, dense neural network. The only benefit I have found is that it makes long vectors more human-readable and creates nice graphs.

\section{Conclusions for P2Rank}

Using the same data as the original P2Rank paper, I have been able to reproduce the simple \textit{P2Rank RFC}. This gave me a baseline, over which I was able to improve the model's performance by including the surroundings for each prediction. The improvement in F1 score from 55 \% to 60 \% is significant, but because of the quadratic complexity of surroundings calculation, using such a model in the production of \cite{prankweb} would bring a significant slowdown and the performance improvement might not be so big as the rest of the pipeline works with the surroundings of the \ac{SAS} point. This is a good starting point for additional research to improve the performance.
\chapter{Code and data availability and usage}

A software framework was developed to run these experiments. This code can be accessed on \href{https://github.com/eleniel-mocna/Refined}{the project GitHub repository}. In this chapter, I'll describe the individual run configurations. Because of the time and memory requirements, there is no single configuration to run all experiments, which would be impractical. Instead, I'll describe the individual run configurations so that individual experiments can be reproduced.

\section{Configuration}

The configuration for this pipeline is governed by the \texttt{config.json} file. Here, the main parameters can be adjusted. The most important

\section{Data}

Each dataset for 

\chapter*{Conclusion}
\addcontentsline{toc}{chapter}{Conclusion}
In this work, I explored the potential of convolutional neural networks (CNNs) for detecting protein-ligand binding sites, specifically focusing on the REFINED algorithm's ability to transform high-dimensional feature vectors into images suitable for CNN analysis. While previous research suggested promising results with this approach, the findings of this thesis did not provide significant evidence to support the superiority of REFINED over established methods.

The experiments conducted on the CHEN11 and COACH420 datasets revealed that:
\begin{itemize}
    \item \textbf{REFINED} did not show a significant improvement in predictive power compared to state-of-the-art approaches, such as Random Forest Classifiers or dense neural networks.*While REFINED successfully rearranged features into images with visual gradients, this did not translate into a demonstrable advantage in binding site prediction accuracy.
    \item \textbf{No correlation was found between the REFINED score and the performance of CNN models trained on the resulting images.} This suggests that REFINED's specific arrangement of features does not inherently enhance CNN's ability to identify binding sites.

\end{itemize}

While the primary hypothesis of this thesis was not confirmed, valuable insights were gained:
\begin{itemize}
    \item \textbf{The importance of considering surrounding features for accurate binding site prediction was confirmed.} Models utilizing the surrounding context of each point, such as the RFC Surrounding and NN models, achieved better performance compared to the P2Rank RFC model that solely relied on individual point features.
    \item \textbf{The potential of CNNs in this domain remains open for further exploration.} Future research could investigate alternative methods for feature vector transformation or different CNN architectures that might better capture the spatial relationships relevant to protein-ligand interactions. 

\end{itemize}

In conclusion, although the anticipated benefits of REFINED were not observed in this study, the exploration of CNNs for binding site prediction offered valuable insights into the complexities of this problem. The findings suggest that while considering the surrounding context of each point is crucial, the specific arrangement of features within images may not be the determining factor for achieving optimal performance. Future research should continue to investigate the potential of CNNs and alternative approaches to advance the field of protein-ligand binding site prediction. \footnote{Writing of the conclusion chapter has been assisted by the Gemini 1.5 Ultra. The writing of the whole work has been assisted by \href{https://www.grammarly.com/premium}{Grammarly} writing assistant. All usage of AI tools was conducted in accordance with the Code of Ethics of the Faculty of Mathematics and Physics of Charles University.}


%%% Bibliography
\include{bibliography}

%%% Figures used in the thesis (consider if this is needed)
\listoffigures

%%% Tables used in the thesis (consider if this is needed)
%%% In mathematical theses, it could be better to move the list of tables to the beginning of the thesis.
% \listoftables

%%% Abbreviations used in the thesis, if any, including their explanation
%%% In mathematical theses, it could be better to move the list of abbreviations to the beginning of the thesis.
\chapwithtoc{List of Abbreviations}
\begin{acronym}
    \acro{CNN}{Convolutional Neural Network. A neural network with convolutional layers. A good description of convolutional layers is \href{https://www.tensorflow.org/api_docs/python/tf/nn/conv2d}{in the TensorFlow documentation}.}
    \acro{SAS}{Solvent accessible surface. Area on the surface of the protein. SAS points then represent local spherical 3D neighborhoods that are centered on them. At the same time, they can be seen as potential locations of contact atoms of potential ligands. For more info on SAS points, see \textit{Materials and methods} of \cite{P2RANK}. For a closer description of SAS, see the \href{https://en.wikipedia.org/wiki/Accessible_surface_area}{relevant Wikipedia page}.}
    \acro{GEA}{Genetic evolutionary algorithm. An optimization algorithm simulating Darwinian evolution. In work, the permutation encoding was used as described \href{https://martinpilat.com/en/nature-inspired-algorithms/evolutionary-algorithms-continuous-and-combinatorial-optimization}{by Martin Pilát.}}
    \acro{HCA}{Hill Climbing Algorithm. Shortly describe in the relevant footnote: ~\ref{hca_description}, or the relevant \href{https://en.wikipedia.org/wiki/Hill_climbing}{Wikipedia page}.}
    \acro{RFC}{Random Forest Classifier. A meta classifier made by averaging between multiple decision tree classifiers. Well described on the relevant \href{https://en.wikipedia.org/wiki/Random_forest}{Wikipedia page}.}
    \acro{LBS}{Ligand Binding Site, see ~\ref{LBS}.}
\end{acronym}

%%% Attachments to the bachelor thesis, if any. Each attachment must be
%%% referred to at least once from the text of the thesis. Attachments
%%% are numbered.
%%%
%%% The printed version should preferably contain attachments, which can be
%%% read (additional tables and charts, supplementary text, examples of
%%% program output, etc.). The electronic version is more suited for attachments
%%% which will likely be used in an electronic form rather than read (program
%%% source code, data files, interactive charts, etc.). Electronic attachments
%%% should be uploaded to SIS and optionally also included in the thesis on a~CD/DVD.
%%% Allowed file formats are specified in provision of the rector no. 72/2017.
\appendix
\chapter{Attachments}

\section{Feature list}
\label{feature_list}
This is the list of features used in the models:
\begin{lstlisting}
['chem.hydrophobic', 'chem.hydrophilic', 'chem.hydrophatyIndex',
'chem.aliphatic', 'chem.aromatic', 'chem.sulfur', 'chem.hydroxyl',
'chem.basic', 'chem.acidic', 'chem.amide', 'chem.posCharge',
'chem.negCharge', 'chem.hBondDonor', 'chem.hBondAcceptor',
'chem.hBondDonorAcceptor', 'chem.polar', 'chem.ionizable', 'chem.atoms',
'chem.atomDensity', 'chem.atomC', 'chem.atomO', 'chem.atomN',
'chem.hDonorAtoms', 'chem.hAcceptorAtoms', 'volsite.vsAromatic',
'volsite.vsCation', 'volsite.vsAnion', 'volsite.vsHydrophobic',
'volsite.vsAcceptor', 'volsite.vsDonor', 'protrusion.protrusion',
'bfactor.bfactor', 'xyz.x', 'xyz.y', 'xyz.z', 'atom_table.apRawValids',
'atom_table.apRawInvalids', 'atom_table.atomicHydrophobicity']
\end{lstlisting}
The label is column is named \texttt{'@@class@@'}.

These features are the same as described in \cite{P2RANK}, Additional File, Table 4.

\openright
\end{document}
